\documentclass[t]{beamer}
\usetheme{Copenhagen}
\setbeamertemplate{headline}{} % remove toc from headers
\beamertemplatenavigationsymbolsempty

\usepackage{amsmath, array, tikz, bm, pgfplots, tcolorbox, graphicx, venndiagram, color, colortbl}
\pgfplotsset{compat = 1.16}
\usepgfplotslibrary{statistics}
\usetikzlibrary{trees}

\title{Speed Counting}
\author{}
\date{}

\AtBeginSection[]
{
  \begin{frame}
    \frametitle{Objectives}
    \tableofcontents[currentsection]
  \end{frame}
}

\begin{document}

\begin{frame} 
\maketitle
\end{frame}

\section{Use the Fundamental Counting Rule}

\begin{frame}{Example 1}
For a special at a restaurant, you can choose between 3 appetizers, 4 entrees, and 2 desserts. If you select one item from each category (appetizer, entree, and dessert), how many different meals can you create?	\newline\\	\pause

Let's call the appetizers A, B, and C. \newline \pause
Let's call the entrees D, E, F, and G. \newline \pause
Let's call the desserts H and I.	\newline\\	\pause

With appetizer A: ADH, ADI, AEH, AEI, AFH, AFI, AGH, AGI \newline \pause
With appetizer B: BDH, BDI, BEH, BEI, BFH, BFI, BGH, BGI \newline \pause
With appetizer C: CDH, CDI, CEH, CEI, CFH, CFI, CGH, CGI \newline\\ \pause

For a total of 24 possible different meals.
\end{frame}

\begin{frame}{Fundamental Counting Rule}
If event $A$ can occur in $a$ different ways and event $B$ can occur in $b$ different ways, then the total number of ways both events can occur is $ab$ ways.	\newline\\	\pause

This can be generalized to multiple events, such as those in example 1: $3 \times 4 \times 2 = 24$
\end{frame}


\section{Understand factorial notation}

\begin{frame}{Example 2}
A baseball lineup consists of 9 players. How many different lineups using all 9 players on a team exist?	\newline\\	\pause

Using the Fundamental Counting Rule with 9 positions:	\newline\\	\pause


\begin{tabular}{|c|c|c|c|c|c|c|}
1st pos. &	2nd pos. &	3rd pos. &	$\dots$ & 7th pos.	&	8th pos. &	9th pos.  \\ \hline
\onslide<4->{9} & \onslide<5->{8} & \onslide<6->{7} & \onslide<7->{$\dots$} & \onslide<8->{3} &\onslide<9->{2} & \onslide<10->{1} \\
\end{tabular}
\vspace{10pt}
\onslide<11->{\[9 \times 8 \times 7 \times \cdots \times 2 \times 1 \onslide<12->{= 362,880 \text{ unique lineups}}\]}

\end{frame}

\begin{frame}{Factorial Notation}
Rather than write out all the numbers from 9 to 1 and then multiplying them, mathematicians created {\color{blue}\textbf{factorial notation}} to expedite the process.	

\onslide<2->{\[9! = 9 \times 8 \times 7 \times \cdots \times 3 \times 2 \times 1 = 362,880\]}

\onslide<3->{In general, for a positive integer $n$, \[n! = n(n-1)(n-2) \cdot \cdots \cdot 3(2)(1)\]}
\onslide<4->{with $0! = 1$}
\end{frame}

\begin{frame}{Factorial Growth}
Factorial values grow very quickly:
\begin{align*}
2! &= 2(1) &= 2 \\
3! &= 3(2)(1) &=6 \\
4! &= 4(3)(2)(1) &= 24 \\
5! &= 5(4)(3)(2)(1) &= 120 \\
6! &= 6(5)(4)(3)(2)(1) &= 720	\\
7! &= 7(6)(5)(4)(3)(2)(1) &= 5,040
\end{align*}
\end{frame}

\begin{frame}{Example 3}
How many ways are there to arrange 5 books on a shelf?
\onslide<2->{\[5! = 120 \text{ different arrangements}\]}
\end{frame}

\section{Find permutations of objects}

\begin{frame}{Example 4}
Five people are competing for three prizes: \$1,000, \$500, and \$100. How many different ways can the prizes be awarded?	\newline\\
\begin{center}
\onslide<2->{
	\begin{tabular}{|c|c|c|}
	\$1,000 		&		\$500			&	\$100			\\	\hline
	\onslide<3->{5}	&	\onslide<4->{4}		&	\onslide<5->{3}	\\
	\end{tabular}
}
\end{center}
\onslide<6->{Using the Fundamental Counting Rule:
			\[ 5 \times 4 \times 3 = 60\text{ different ways}\]}
\end{frame}

\begin{frame}{Takeaways from Example 4}
We had more contestants available to win prizes than we had prizes available. We could have had an equal number of contestants and prizes, but we can't have more prizes available than contestants. \newline\\	\pause

If we had 10,000 contestants and 75 prizes, we would have a lot of multiplying to do.	\newline\\	\pause

So is there an easy way to do this if that's the case?	\newline\\	\pause

Yes, and that is where {\color{blue}\textbf{permutations}} come into play.
\end{frame}

\begin{frame}{Example 5}
How many ways are there to award gold, silver, and bronze medals to 8 contestants?	\newline\\
\begin{center}
\onslide<2->{
	\begin{tabular}{|c|c|c|}
	Gold 		&		Silver			&	Bronze			\\	\hline
	\onslide<3->{8}	&	\onslide<4->{7}		&	\onslide<5->{6}	\\
	\end{tabular}
}
\end{center}
\onslide<6->{Using the Fundamental Counting Rule:
			\[ 8 \times 7 \times 6 = 336\text{ different ways}\]}
\end{frame}

\begin{frame}{Example 5}
\[\alert{8} \times \alert{7} \times \alert{6} \times \dots \times 2 \times 1\]
\end{frame}

\begin{frame}{Example 5}
\[\frac{\alert{8} \times \alert{7} \times \alert{6} \times 5 \times 4 \times 3 \times 2 \times 1}{5 \times 4 \times 3 \times 2 \times 1}\]
\end{frame}

\begin{frame}{Example 5}
\[\alert{8} \times \alert{7} \times \alert{6} = 336\]
\end{frame}

\begin{frame}{Permutations}
If there are $n$ items available and we take $r$ at a time, then the total number of permutations is given by 
\[_nP_r = \frac{n!}{(n-r)!}\]

with $n \geq r$	\newline\\

\pause With permutations, the order in which an item is selected matters.
\end{frame}

\begin{frame}{Knowing to Use Permutations}
Permutations
\begin{itemize}
	\item<+-> Offering various prizes
	\item<+-> Running a race
	\item<+-> Assigning officer positions
	\item<+-> Combination locks and passwords
\end{itemize}
\end{frame}

\begin{frame}{Example 6}
How many ways are there of selecting a president, vice president, secretary, and treasurer out of a pool of 10 candidates?	\newline\\	\pause
Order matters.	\newline	\pause
We have $n = 10$ items to choose from.	\newline	\pause
We are filling $r = 4$ positions at a time.	\newline	
\begin{align*}
\onslide<5->{_{10}P_4 &= \frac{10!}{(10-4)!}} \\[8pt]
\onslide<6->{&= \frac{10!}{6!}} \\[8pt]
\onslide<7->{&= 5,040}
\end{align*}
\end{frame}

\section{Find combinations of objects}

\begin{frame}{Combinations}
With permutations, order selection mattered, so 
\begin{center}
ABC, ACB, BAC, BCA, CAB, and CBA
\end{center}
were all different.	\newline\\	\pause

With combinations, selection order does not matter, so there is no distinction among the orderings above. So,
\begin{center}
ABC, ACB, BAC, BCA, CAB, and CBA
\end{center}
are all the same.
\end{frame}

\begin{frame}{Combinations}
Notice that there are 6, or $3!$, arrangements of the letters A, B, and C. \newline\\	\pause

This can help us develop the formula for finding the number of combinations of $n$ items taken $r$ at a time.
\end{frame}

\begin{frame}{Example 7}
Five people are competing for three equal prizes. How many ways can the prizes be awarded?	\newline\\

If order mattered, there would be $_5P_3 = 60$ different possibilities:
\begin{center}
\begin{tabular}{ccccccc}
ABC & ABD & ABE & ACB & ACD & ACE & $\dots$ \\
BAC & BAD & BAE & BCA & BCD & BCE & $\dots$ \\
$\vdots$ & $\vdots$ & $\vdots$ & $\vdots$ & $\vdots$ & $\vdots$ & $\dots$ 
\end{tabular}
\end{center}
\end{frame}

\begin{frame}{Example 7}
Since ABC is the same as ACB, BAC, BCA, CAB, and CBA in the eyes of combinations, we can divide our permutation result of 60 by 3! to get \alert{10 combinations}:

\begin{center}
\begin{tabular}{ccccc}
ABC & ABD & ABE & ACD & ACE \\
ADE & BCD & BCE & BDE & CDE \\
\end{tabular}
\end{center}
\end{frame}

\begin{frame}{Combinations}
If we have $n$ items available and we take $r$ at a time {\color{blue}\textbf{without regard to order of selection}}, then the total number of possible combinations are

\onslide<2->{\[_nC_r = \frac{n!}{r!(n-r)!} \onslide<3->{= \binom{n}{r}}\]}

\onslide<4->{
	Combinations
	\begin{itemize}
		\item<5-> Awarding equal prizes
		\item<6-> Combinations (not the lock though)
		\item<7-> Committees
	\end{itemize}
}
\end{frame}

\begin{frame}{Example 8}
A committee of 5 is to be formed from a pool of 12 potential candidates. How many committees are possible?	\newline\\	
\onslide<2->{We have $n=12$ candidates to choose from} \newline
\onslide<3->{We are selecting $r=5$ at a time}
\begin{align*}
\onslide<4->{_{12}C_5 &= \frac{12!}{5!(12-5)!}}	\\[10pt]
\onslide<5->{&= 792}
\end{align*}
\end{frame}

\begin{frame}{Example 9}
A committee of 5 is to be formed from a pool of 12 potential candidates. The committee is to be made up of 3 managers and 2 accountants. If there are 8 managers and 4 accountants available, how many committees can be formed?	\newline\\
\onslide<2->{We need to make sure 3 of the positions are managers and 2 of the positions are accountants.} \newline\\
\onslide<3->{Thus, we must blend the Fundamental Counting Rule with combinations.}	
\end{frame}

\begin{frame}{Example 9}
\begin{center}
\setlength{\extrarowheight}{6pt}
\begin{tabular}{ccc}
Select 3 managers from 8 & $\times$ & Select 2 accountants from 4 \\ \hline
\onslide<2->{$_8C_3$} & $\times$ & \onslide<3->{$_4C_2$} \\
\onslide<4->{56} & $\times$ & \onslide<5->{6} \\
\end{tabular}
\end{center}
\onslide<6->{There are 336 ways to do this.}
\end{frame}

\begin{frame}{Example 10}
Starting at point $A$ and only moving right or up, how many paths are there to get to point $B$?	
\begin{center}
\begin{tikzpicture}

\end{tikzpicture}
\end{center}
\end{frame}

\section{Find probabilities using counting techniques}

\end{document}