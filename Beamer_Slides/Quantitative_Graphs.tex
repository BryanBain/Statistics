\documentclass[t]{beamer}
\usetheme{Copenhagen}
\setbeamertemplate{headline}{} % remove toc from headers
\beamertemplatenavigationsymbolsempty

\usepackage{amsmath, array, tikz, bm, pgfplots, tcolorbox, graphicx, venndiagram, color, colortbl}
\pgfplotsset{compat = 1.16}
\usepgfplotslibrary{statistics}
\usetikzlibrary{calc}

\title{Quantitative Graphs}
\author{}
\date{}

\AtBeginSection[]
{
  \begin{frame}
    \frametitle{Objectives}
    \tableofcontents[currentsection]
  \end{frame}
}

\begin{document}

\begin{frame} 
\maketitle
\end{frame}

\begin{frame}{Frequency Distribution for Quantitative Data}
The weights (in pounds) of 25 husky dogs are shown below:
\begin{center}
\begin{tabular}{ccccc}
53 & 46 & 44 & 47 & 50 \\
49 & 47 & 44 & 61 & 44 \\
35 & 46 & 49 & 51 & 48 \\
50 & 52 & 44 & 50 & 47 \\
58 & 47 & 52 & 37 & 54 \\
\end{tabular}
\end{center}
Suppose we want to create a frequency distribution for the weights of these awesome dogs. \newline\\	\pause

Since this data is quantitative, we are going to have to decide what each of our ranges of weights in our classes is going to be. 
\end{frame}

\begin{frame}{Definitions for Quantitative Data}
The smallest value (weight in our case) in each class (table row) is called the {\color{blue}\textbf{lower class limit}}. \newline\\	\pause

The largest value in each class is the {\color{blue}\textbf{upper class limit}}. \newline\\ \pause 

Typically, the closer the lower and upper class limits are in value, the more classes we will need. \newline\\	\pause

The difference between two consecutive lower class limits is called the {\color{blue}\textbf{class width}}.	\newline\\	\pause

Let's create a frequency distribution for the dog weights using a class width of 5 pounds.
\end{frame}

\begin{frame}{Frequency Distribution of the Weights of Adorable Huskies}
\begin{center}
\begin{tabular}{c|c}
\textbf{Weight} & \textbf{Frequency} \\ \hline
35 -- 39 & 2 \\
40 -- 44 & 4 \\
45 -- 49 & 9 \\
50 -- 54 & 8 \\
55 -- 59 & 1 \\
60 -- 64 & 1 \\
\end{tabular}
\end{center}
\end{frame}

\end{document}