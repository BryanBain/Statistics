\documentclass[t]{beamer}
\usetheme{Copenhagen}
\setbeamertemplate{headline}{} % remove toc from headers
\beamertemplatenavigationsymbolsempty

\usepackage{amsmath, array, tikz, bm, pgfplots, tcolorbox, graphicx, venndiagram, color, colortbl, xfrac}
\pgfplotsset{compat = 1.16}
\usepgfplotslibrary{statistics}
\usetikzlibrary{calc}

\title{Hypothesis Testing}
\subtitle{or: How I Learned to Stop Worrying and Love Inferential Statistics}
\author{}
\date{}

\AtBeginSection[]
{
  \begin{frame}
    \frametitle{Objectives}
    \tableofcontents[currentsection]
  \end{frame}
}

\begin{document}

\pgfmathdeclarefunction{gauss}{2}{%
  \pgfmathparse{1/(#2*sqrt(2*pi))*exp(-((x-#1)^2)/(2*#2^2))}%
}

\begin{frame} 
\maketitle
\end{frame}

\section{State the null and alternative hypothesis}

\begin{frame}{What is Hypothesis Testing?}
\textbf{Hypothesis testing} is the process of testing a claim about a (population) parameter by obtaining a sample and analyzing the results against that parameter.	\newline\\	\pause

\begin{tcolorbox}[colframe=green!20!black, colback = green!30!white,title=\textbf{Null Hypothesis}]
The \textbf{null hypothesis} is the original claim about the parameter, denoted $H_0$, and usually is stated in terms of equality.
\end{tcolorbox}
\bigskip	\pause

\begin{tcolorbox}[colframe=green!20!black, colback = green!30!white,title=\textbf{Alternative Hypothesis}]
The \textbf{alternative hypothesis}, denoted $H_A$, is the new claim that is made against the null hypothesis.
\end{tcolorbox}
\end{frame}

\begin{frame}{Types of Hypothesis Testing}
\begin{itemize}
	\item Left-tailed
	\begin{itemize}
		\item<2->{$H_A$: parameter $ < H_0$}
\end{itemize}	\bigskip
	\item<3-> Right-tailed
	\begin{itemize}
		\item<4->{$H_A$: parameter $ > H_0$}
	\end{itemize}	\bigskip
	\item<5-> Two-tailed
	\begin{itemize}
		\item<6->{$H_A$: parameter $\neq H_0$}
	\end{itemize}
\end{itemize}
\end{frame}

\begin{frame}{Example 1}
Determine the null and alternative hypotheses for each.	\newline\\
(a) \quad A new car's mpg is listed as 33. You want to know if the mpg is not 33.	\newline\\	\pause

$H_0: \, \mu = 33\text{ mpg}$	\newline\\	\pause
$H_A: \, \mu \neq 33\text{mpg}$ 
\end{frame}

\begin{frame}{Example 1}
(b) \quad A hospital says the mean wait time for patients to see a doctor is 3.5 minutes. You want to know if the mean wait time is more than 3.5 minutes.	\newline\\	\pause

$H_0: \, \mu = 3.5\text{ minutes}$ \newline\\	\pause
$H_A: \, \mu > 3.5\text{ minutes}$
\end{frame}

\begin{frame}{Example 1}
(c) \quad A company claims their program will increase your grade in statistics class by 10\%. You think it might not be that much.	\newline\\	\pause

$H_0: \, \mu = 0.10$ \newline\\	\pause
$H_A: \, \mu < 0.10$
\end{frame}

\section{Understand errors and interpret \textit{p}-value}

\begin{frame}{What Do Our Sample Results Mean?}
Our sample results will allow us to do one of two things:	\newline\\
\begin{itemize}
	\item<2->{Reject the null hypothesis} \newline\\
	\item<3->{Fail to reject the null hypothesis}
\end{itemize}
\end{frame}

\begin{frame}{Rejecting the Null Hypothesis}
If our sample statistics give us reason to believe that the null hypothesis may not in fact be true, then we will state our rejection of it.	\newline\\	\pause

\begin{quote}
... we have sufficient evidence to reject [the null hypothesis].
\end{quote}	\bigskip		\pause

Rejecting the null hypothesis is like a jury declaring a defendant guilty. \newline\\	\pause

\emph{Note:} There is still a chance that the defendant is innocent, but the evidence is strong enough to bring a guilty verdict.
\end{frame}

\begin{frame}{Failing to Reject the Null Hypothesis}
However, our sample statistics might not give us reason to believe the null hypothesis is false. \newline\\	\pause

\begin{quote}
... we do not have sufficient evidence to reject [the null hypothesis].
\end{quote}	\bigskip \pause

Failing to reject the null hypothesis is like a jury declaring a defendant not guilty.	\newline\\	\pause

\emph{Note:} A declaration of not guilty \underline{is not the same as} a declaration of innocence. There just is not sufficient evidence to declare guilt, and the defendant \emph{could still actually be guilty}.
\end{frame}

\begin{frame}{Errors}
Just like in courts of law, sometimes innocent people are declared guilty and guilty people are declared not guilty. The same can happen in statistics.	\newline\\	\pause

\begin{tcolorbox}[colframe=green!20!black, colback = green!30!white,title={\textbf{Type I Error}}]
A \textbf{Type I error} in statistics occurs when we reject a null hypothesis that is actually true.
\end{tcolorbox}
\bigskip \pause

The probability of making a Type I error is $\alpha$, which we saw earlier with our confidence intervals in the form $1-\alpha$.	\newline\\	\pause

In hypothesis testing, $\alpha$ is called the \textbf{level of significance}.
\end{frame}

\begin{frame}{Errors}
\begin{tcolorbox}[colframe=green!20!black, colback = green!30!white,title={\textbf{Type II Error}}]
A \textbf{Type II error} in statistics occurs when we fail to reject a null hypothesis that is actually false.
\end{tcolorbox}
\bigskip	\pause

The probability of making a Type II error is $\beta$.	\newline\\ \pause

\emph{Note:} As we decrease the chances of making a Type I error, we increase the chances of making a Type II error. \newline\\	\pause

If we make it harder to put an innocent person in jail, we make it tougher to return a guilty verdict. This will also have the affect of increasing the number of actual guilty defendants who are let go.	\newline\\	\pause

The \textbf{power} of a test is given as $1 - P(\beta)$
\end{frame}

\begin{frame}{Errors Summary}
\begin{center}
\begin{tabular}{c|cc}
$\bm{H_0}$ & Reject $H_0$ & Fail to reject $H_0$ \\ \hline
$H_0$ True & Type I error & Correct decision \\
$H_0$ False & Correct decision & Type II error \\
\end{tabular} 
\\[1cm]
\begin{tabular}{c|cc}
\textbf{Defendant} & Declare Guilty & Declare Not Guilty \\ \hline
Actually Innocent & Type I error & Correct decision \\
Actually Guilty & Correct decision & Type II error \\
\end{tabular}
\end{center}
\end{frame}

\begin{frame}{$p$-Value}
Recall that {\color{blue}\textbf{statistically significant}} means that something is not likely to occur by chance. \newline\\	\pause

\begin{tcolorbox}[colframe=green!20!black, colback = green!30!white,title=\textbf{\textit{p}-Value}]
The \textbf{\textit{p}-value} is the probability of obtaining a sample as extreme as the one obtained \textbf{under the assumption that the null hypothesis is true.}
\end{tcolorbox}
\bigskip \pause

If our $p$-value is less than a given acceptable value ($\alpha$), then our sample was not likely to occur by chance \emph{assuming the null hypothesis is true}, \pause so we have sufficient evidence to reject the null hypothesis.
\end{frame}

\begin{frame}{Example 2}

\end{frame}

% p-value
% errors
% hypothesis testing of the population mean with known population st. dev.
  % critical value method
  % confidence interval method
  % p-value method

\end{document}