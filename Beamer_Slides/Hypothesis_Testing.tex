\documentclass[t]{beamer}
\usetheme{Copenhagen}
\setbeamertemplate{headline}{} % remove toc from headers
\beamertemplatenavigationsymbolsempty

\usepackage{amsmath, array, tikz, bm, pgfplots, tcolorbox, graphicx, venndiagram, color, colortbl, xfrac}
\pgfplotsset{compat = 1.16}
\usepgfplotslibrary{statistics}
\usetikzlibrary{calc}

\title{Hypothesis Testing}
\subtitle{or: How I Learned to Stop Worrying and Love Inferential Statistics}
\author{}
\date{}

\AtBeginSection[]
{
  \begin{frame}
    \frametitle{Objectives}
    \tableofcontents[currentsection]
  \end{frame}
}

\begin{document}

\pgfmathdeclarefunction{gauss}{2}{%
  \pgfmathparse{1/(#2*sqrt(2*pi))*exp(-((x-#1)^2)/(2*#2^2))}%
}

\begin{frame} 
\maketitle
\end{frame}

\section{State the null and alternative hypothesis}

\begin{frame}{What is Hypothesis Testing?}
\textbf{Hypothesis testing} is the process of testing a claim about a (population) parameter by obtaining a sample and analyzing the results against that parameter.	\newline\\	\pause

\begin{tcolorbox}[colframe=green!20!black, colback = green!30!white,title=\textbf{Null Hypothesis}]
The \textbf{null hypothesis} is the original claim about the parameter, denoted $H_0$, and usually is stated in terms of equality.
\end{tcolorbox}
\bigskip	\pause

\begin{tcolorbox}[colframe=green!20!black, colback = green!30!white,title=\textbf{Alternative Hypothesis}]
The \textbf{alternative hypothesis}, denoted $H_A$, is the new claim that is made against the null hypothesis.
\end{tcolorbox}
\end{frame}

\begin{frame}{Types of Hypothesis Testing}
\begin{itemize}
	\item Left-tailed
	\begin{itemize}
		\item<2->{$H_A$: parameter $ < H_0$}
\end{itemize}	\bigskip
	\item<3-> Right-tailed
	\begin{itemize}
		\item<4->{$H_A$: parameter $ > H_0$}
	\end{itemize}	\bigskip
	\item<5-> Two-tailed
	\begin{itemize}
		\item<6->{$H_A$: parameter $\neq H_0$}
	\end{itemize}
\end{itemize}
\end{frame}

\begin{frame}{Example 1}
Determine the null and alternative hypotheses for each.	\newline\\
(a) \quad 
\end{frame}

\begin{frame}{$p$-Value}
Recall that {\color{blue}\textbf{statistically significant}} means that something is not likely to occur by chance. \newline\\	\pause

\end{frame}

% p-value
% errors
% hypothesis testing of the population mean with known population st. dev.
  % critical value method
  % confidence interval method
  % p-value method

\end{document}