\documentclass[t]{beamer}
\usetheme{Copenhagen}
\setbeamertemplate{headline}{} % remove toc from headers
\beamertemplatenavigationsymbolsempty

\usepackage{amsmath, array, tikz, bm, pgfplots, tcolorbox, graphicx, venndiagram, color, colortbl, xfrac}
\pgfplotsset{compat = 1.16}
\usepgfplotslibrary{statistics}
\usetikzlibrary{calc}

\title{Hypothesis Testing}
\subtitle{Two Sample Means}
\author{}
\date{}

\AtBeginSection[]
{
  \begin{frame}
    \frametitle{Objectives}
    \tableofcontents[currentsection]
  \end{frame}
}

\begin{document}

\pgfmathdeclarefunction{gauss}{2}{%
  \pgfmathparse{1/(#2*sqrt(2*pi))*exp(-((x-#1)^2)/(2*#2^2))}%
}

\begin{frame} 
\maketitle
\end{frame}

\begin{frame}{Moving from One Sample to Two}
In this section, we are going to analyze the difference of means between two samples of data. \newline\\	\pause

In particular, we will examine whether there is a difference in means that is statistically significant, assuming there is no difference in means of each sample's population. \newline\\	\pause

To put it mathematically, if we denote our first sample as $X$, and our second sample as $Y$, then we will want to know if
\[\mu_X - \mu_Y = 0 \pause \quad \text{or} \quad \mu_X = \mu_Y\]
\pause
Thus, our null hypotheses will be $\mu_X = \mu_Y$.
\end{frame}

\section{Perform hypothesis test on the mean for two dependent samples}

\begin{frame}{Dependent Samples}
Recall from probability that two events are {\color{blue}\textbf{dependent}} if the chance of the second event happening is affected by the first event happening.	\newline\\	\pause

For this part of the section, we will look at examples with data that is paired together, often times having somewhat of a ``before-and-after" theme; much like with experiments.	\newline\\	\pause

This part is sometimes known as a {\color{blue}\textbf{paired $\bm{t}$ test}} or a {\color{blue}\textbf{matched pairs test}}.
\end{frame}

\begin{frame}{Nuts and Bolts of a Paired $t$ Test}
We will assume the distribution in the differences of sample $X$ and sample $Y$ are normal.	\newline\\	\pause

The test statistic is
\[t = \frac{\overline{d} - 0}{s_d / \sqrt{n}} \pause = 	\frac{\overline{d}}{s_d / \sqrt{n}} \]		\pause

$\overline{d}$ represents the mean differences between the samples. \[\overline{d} = \overline{X-Y}\]		\pause

The population difference, $\mu_d$ will be 0 for hypothesis testing.	\newline\\	\pause

$s$ represents the sample standard deviation of the differences, and there are $n-1$ degrees of freedom.
\end{frame}

\begin{frame}{Example 1}
A medication is given to patients in an attempt to lower their LDL cholesterol. The tables below list the levels. At the 5\% significance level, test the claim that the medicine is effective in lowering LDL cholesterol.	\newline\\
\begin{tabular}{c|c}
\textbf{Before} & \textbf{After} \\ \hline
95 & 91 \\
109 & 107 \\
127 & 129 \\
131 & 125 \\
117 & 110 \\
135 & 120 \\
103 & 97 \\
98 & 101 \\
111 & 107 \\
\end{tabular}
\end{frame}

\begin{frame}{Example 1}
$H_0: \, \mu_{\text{Before}} = \mu_{\text{After}}$	\newline\\
$H_A: \, \mu_{\text{Before}} < \mu_{\text{After}}$	\newline\\	\pause
\begin{itemize}
	\item $t = 2.446$ \quad \pause (critical value = 1.96)	\newline\\	\pause
	\item $p = 0.0201$ \quad \pause ($\alpha = 0.05$) \newline\\	\pause
	\item 95\% confidence interval: (0.2478, 8.4189) \quad \pause does not contain $\mu_d = 0$
\end{itemize}
\end{frame}

\begin{frame}{Example 1}
Reject the null hypothesis.		\newline\\	\pause
\begin{quote}
At the 5\% significance level we reject the null hypothesis that there is no difference in LDL cholesterol levels and conclude that our sample suggests the medication may be effective in lowering LDL cholesterol levels.
\end{quote}
\end{frame}

\section{Perform hypothesis test on the mean for two independent samples}

\begin{frame}{Independent Samples}
For this part, our samples will be \emph{independent} of one another.	\newline\\	\pause

In other words, there is no ``before-and-after" relationship between our samples, and our samples don't even have to be the same sizes.
\end{frame}

\begin{frame}{To Pool or Not To Pool}
Since our independent samples don't even have to have the same sample size, we will have to know about assumptions we can make about their \emph{variances}. \newline\\	\pause

If we operate under the assumption that the two populations have equal variances, $\sigma_X = \sigma_Y$, then we use a \textbf{pooled} (i.e. \textit{weighted}) estimate of the common variance. \newline\\	\pause

However, $p$-values obtained through pooled tests can be significantly off if the population variances are, in fact, \emph{not} equal. \newline\\	\pause

As such, the examples in this section will assume that the population variances are \emph{not equal}.
\end{frame}

\begin{frame}{Nuts and Bolts}
\textbf{Assumptions}
\begin{itemize}
\item<2-> Two samples are randomly chosen and independent of each other.
\item<3-> Population distributions are approximately normal, OR
\item<4-> Sample size are large enough ($n_1 \geq 30$ and $n_2 \geq 30$)
\end{itemize}
\bigskip
\onslide<5->{Test statistic is given as}
\onslide<6->{\[t = \frac{\overline{d}}{\sqrt{\frac{s_1^2}{n_1}+\frac{s_2^2}{n_2}}}\]}
\end{frame}

\begin{frame}{Degrees of Freedom}
The degrees of freedom is found by calculating
\[\text{df} = \dfrac{\left(V_1+V_2\right)^2}{\dfrac{V_1^2}{n_1-1} + \dfrac{V_2^2}{n_2-1}}\]
\newline\\
where
\[V_1 = \frac{s_1^2}{n_1} \text{ and } V_2 = \frac{s_2^2}{n_2}\]
\newline\\
\emph{Note}: Round the degrees of freedom down to the nearest integer.
\end{frame}

\begin{frame}{For Pools and Giggles}
If working with pooled variances,
\[t = \frac{\overline{d}}{\sqrt{s_p^2\left(\frac{1}{n_1}+\frac{1}{n_2}\right)}}\]
\bigskip
\[s_p^2 = \frac{(n_1-1)s_1^2+(n_2-1)s_2^2}{n_1+n_2-2}\]
\bigskip
\[\text{df} = n_1+n_2-2\]
\end{frame}
% Pooled vs. Non-pooled
\end{document}