\documentclass[t]{beamer}
\usetheme{Copenhagen}
\setbeamertemplate{headline}{} % remove toc from headers
\beamertemplatenavigationsymbolsempty

\usepackage{amsmath, array, tikz, bm, pgfplots, tcolorbox, graphicx, venndiagram, color, colortbl, xfrac}
\pgfplotsset{compat = 1.16}
\usepgfplotslibrary{statistics}
\usetikzlibrary{calc}

\title{Hypothesis Testing}
\subtitle{Two Sample Proportions}
\author{}
\date{}

\AtBeginSection[]
{
  \begin{frame}
    \frametitle{Objectives}
    \tableofcontents[currentsection]
  \end{frame}
}

\begin{document}

\pgfmathdeclarefunction{gauss}{2}{%
  \pgfmathparse{1/(#2*sqrt(2*pi))*exp(-((x-#1)^2)/(2*#2^2))}%
}

\begin{frame} 
\maketitle
\end{frame}

\begin{frame}{Second Verse Same(?) as the First}
In this section we will be testing the differences between two proportions, much like we did with testing two means. \newline\\	\pause

Much of the material will be the two-sample version of hypothesis testing of a single proportion.
\end{frame}

\begin{frame}{Test Statistic and $p$-Value}
The test statistic for two sample proportions is given by
\[
t = \frac{(\hat{p}_1-\hat{p}_2) - (p_1 - p_2)}{\sqrt{\frac{\hat{p}_1(1-\hat{p}_1)}{n_1}+\frac{\hat{p}_2(1-\hat{p}_2)}{n_2}}}
\]
where 	\pause
\begin{itemize}
	\item $\hat{p}_1 \text{ and } \hat{p}_2$ represent the sample proportions	\pause
	\item $p_1 \text{ and } p_2$ represent the population proportions	\pause
	\item $n_1 \text{ and } n_2$ represent the sample sizes	\pause
\end{itemize}

The $p$-value is found in the same manner as other sections.
\end{frame}

\begin{frame}{Just For Pools and Giggles}
If using pooled variances, use
\[
\hat{p} = \frac{x_1 + x_2}{n_1 + n_2}
\]
\pause	\newline\\
The test statistic then becomes
\[
t = \frac{(\hat{p}_1-\hat{p}_2) - (p_1 - p_2)}{\sqrt{\hat{p}(1-\hat{p})\left(\frac{1}{n_1} + \frac{1}{n_2}\right)}}
\]
\end{frame}

\begin{frame}{Confidence Intervals}
The $1-\alpha$ confidence interval is calculated as
\[
(\hat{p}_1 - \hat{p}_2) \pm z_{\alpha/2}\sqrt{\frac{\hat{p}_1(1-\hat{p}_1)}{n_1}+\frac{\hat{p}_2(1-\hat{p}_2)}{n_2}}
\]	\pause
Remember, if our confidence interval contains the claimed difference in population proportions, we do not reject the null hypothesis.
\end{frame}

\begin{frame}{You Know What Happens When You Assume, Don't You?}
\textbf{Assumptions for this section}:
\begin{itemize}
	\item<2-> Samples are randomly selected
	\item<3-> Samples are independent
	\item<4-> Each sample size is large enough so that the differences in sampling proportions is approximately normal
	\begin{itemize}
	\item<5-> Good to go if $n_1\hat{p}_1$, $n_1(1-\hat{p}_1)$, $n_2\hat{p}_2$, and $n_2(1-\hat{p}_2)$ are each greater than 5.
	\end{itemize}
\end{itemize}
\end{frame}

\begin{frame}{Example 1}
A poll of 450 registered voters is taken and 43\% of them would vote for the incumbent candidate. A week later a poll of 300 different registered voters is taken and 41\% of them would vote for the incumbent candidate. \newline\\

At the $\alpha = 0.05$ significance level, test the claim that the proportions of all registered voters would vote for the incumbent candidate is now different.	\newline\\	\pause

$H_0: \, p_1 = p_2$ \newline	\pause
$H_A: \, p_1 \neq p_2$
\end{frame}

\begin{frame}{Example 2}
According to recent hospital data, 44 out of 175 recent men were hospitalized for heart conditions, while 21 out of 107 women were. At the $\alpha = 0.05$ level of significance, test the claim that the proportion of men admitted for heart conditions is higher than that of women.	\newline\\	\pause

$H_0: \, p_{\text{men}} = p_{\text{women}}$ \newline \pause
$H_A: \, p_{\text{men}} > p_{\text{women}}$
\end{frame}
\end{document}