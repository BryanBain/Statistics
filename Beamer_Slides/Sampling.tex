\documentclass[t]{beamer}
\usetheme{Copenhagen}
\setbeamertemplate{headline}{} % remove toc from headers
\beamertemplatenavigationsymbolsempty

\usepackage{amsmath, array, tikz, bm, pgfplots, tcolorbox, graphicx, venndiagram, color, colortbl}
\pgfplotsset{compat = 1.16}
\usepgfplotslibrary{statistics}
\usetikzlibrary{trees}

\title{Sampling}
\author{}
\date{}

\AtBeginSection[]
{
  \begin{frame}
    \frametitle{Objectives}
    \tableofcontents[currentsection]
  \end{frame}
}

\begin{document}

\begin{frame} 
\maketitle
\end{frame}

\section{Classify a data collection method as an observational study or an experiment}

\begin{frame}{Observational Study}
\begin{tcolorbox}[colframe=green!20!black, colback = green!30!white,title=\textbf{Observational Study}]
An \textbf{observational study} is a method to obtain data in which the collector (researcher) does not get involved.
\end{tcolorbox}
\vspace{8pt} \pause

Observational studies try to take as much of a \textit{hands-off approach} as possible. \newline\\	\pause

In other words, the researcher observes behaviors and takes notes, but does not interject themselves into the study. 
\end{frame}

\begin{frame}{Experiment}
\begin{tcolorbox}[colframe=green!20!black, colback = green!30!white,title=\textbf{Experiment}]
In an \textbf{experiment}, the researcher divides the subjects into groups, applies a treatment to one group, and notes the effects (if any) between the groups.
\end{tcolorbox}
\vspace{8pt}	\pause
Typically divided into 2 groups:
\begin{itemize}
	\item \textbf{Experimental}: group that receives the treatment.	\pause
	\item \textbf{Control}: group that either does not receive treatment or receives a ``fake" treatment (such as a \textit{placebo}).
\end{itemize}
\end{frame}

\begin{frame}{Example 1}
Classify each as either an observational study or an experiment.	\newline\\	\pause
(a)	\quad A survey of 1,000 people is conducted to determine the best dog breeed. \newline\\	\pause
Observational study \newline\\ \pause
(b)	\quad 83 patients are given a new anxiety medication and 75 patients are given a sugar pill.	\newline\\	\pause
Experiment
\end{frame}

\section{Examine various sampling methods}

\begin{frame}{Sampling Methods}
There are various ways in which to take samples, and depending on the research, one might be more appropriate than another.	\newline\\	\pause

However, keep in mind that {\color{blue}\textbf{good sampling incorporates randomness into the process.}}
\end{frame}

\begin{frame}{Sampling Methods}
\begin{itemize}
	\item<+-> \textbf{Simple random sampling}
	\begin{itemize}
		\item<+-> Each member of the population has an equal chance of being selected.
	\end{itemize}	\vspace{10pt}
	\item<+-> \textbf{Stratified sampling}
	\begin{itemize}
		\item<+-> Divide the population into non-overlapping groups (\textit{strata}).
		\item<+-> Randomly sample from each group.
		\item<+-> (Some from all)
	\end{itemize}
\end{itemize}
\end{frame}

\begin{frame}{Sampling Methods}
\begin{itemize}
	\item<+-> \textbf{Cluster sampling}
	\begin{itemize}
		\item<+-> Divide the population into non-overlapping groups.
		\item<+-> Randomly pick groups and obtain information from everyone in those groups.
		\item<+-> (All from some)
	\end{itemize}	\vspace{10pt}
	\item<+-> \textbf{Systematic sampling}
	\begin{itemize}
		\item<+-> Subjects are placed in some order.
		\item<+-> Pick a random starting value ($n$).
		\item<+-> Pick a random value to count by ($k$).
		\item<+-> Starting at $n$, take every $k$\textsuperscript{th} subject thereafter.
	\end{itemize}
\end{itemize}
\end{frame}

\begin{frame}{Sampling Methods}
\begin{itemize}
	\item<+-> \textbf{Convenience sampling}
	\begin{itemize}
		\item<+-> Subjects select themselves.
		\item<+-> Results are easily obtained.
		\item<+-> Least effective and desirable method.
		\item<+-> Also known as a \textit{volunary response sample}.
	\end{itemize}
\end{itemize}
\end{frame}

\begin{frame}{Example 2}
Classify each by the sampling method used.	\newline\\
(a) \quad An achievement test is given to all 9\textsuperscript{th} and 12\textsuperscript{th} grade students at a local high school.	\quad	\pause
\alert{Cluster sampling} \newline\\	\pause
(b) \quad A radio station asks listeners to call in with their opinion on a political issue.	\quad	\pause
\alert{Convenience sampling} \newline\\ \pause
(c) \quad A quality control manager selects the 5\textsuperscript{th} circuit board on an assembly line and then selects every 14\textsuperscript{th} chip after that.	\quad	\pause
\alert{Systematic sampling} 
\end{frame}

\begin{frame}{Example 2}
(d) \quad 10 seniors, 12 juniors, 13 sophomores, and 10 freshmen are asked to name their favorite food. \quad \pause
\alert{Stratified sampling} \newline\\ \pause
(e) \quad 10 names are drawn out of a hat containing 50 names. \quad \pause
\alert{Simple random sampling}
\end{frame}

\section{Examine various types of observational studies and experiments}

\begin{frame}{Types of Observational Studies}
\begin{itemize}
	\item Cross-Sectional
		\begin{itemize}
		\item<+-> Data is collected and observed at one point in time. 
		\end{itemize}
	\item<+-> Retrospecitve
		\begin{itemize}
		\item<+-> Collecting data from past events 
		\end{itemize}
	\item<+-> Longitudinal
		\begin{itemize}
		\item<+-> Collect future data from groups with common factors.
		\end{itemize}
\end{itemize}
\end{frame}

\begin{frame}{Types of Experiments}
\begin{itemize}
	\item Blind
	\begin{itemize}
		\item<+-> Researcher knows what group (experimental vs. control) the subject is in, but the subject doesn't.
	\end{itemize}
	\item Double-blind
	\begin{itemize}
		\item<+-> Neither the researcher nor the subject knows which group the subject is in; a third party knows but does not reveal. 
	\end{itemize}
\end{itemize}
\end{frame}

\begin{frame}{Designing an Experiment}
Good experiments will take the following into consideration:
\begin{itemize}
	\item<+-> Subjects are assigned to different groups through random selection.
	\item<+-> Experiments are performed on more than one subject (known as \textit{replication}).
	\item<+-> The sample size is large enough to see the true nature of the effects.
	\item<+-> Researchers will control the effects of the variables using such techinques as blinding.
\end{itemize}
\end{frame}

\section{Examine errors and other issues in sampling}

\begin{frame}{Errors}
No matter how well-designed a study or experiment can be, there is always room for error.
\begin{itemize}
	\item<+-> \textbf{Sampling error}
	\begin{itemize}
		\item<+-> The difference between the sample result and the true population result.
	\end{itemize}
	\item<+-> \textbf{Nonsampling error}
	\begin{itemize}
		\item<+-> Sample data isn't collected, recorded, or analyzed correctly.
	\end{itemize}
	\item<+-> \textbf{Not using randomness}
	\begin{itemize}
		\item<+-> Avoid (or take with healthy dose of skepticism) sample data that does not have some component of randomness to it, such as a convenience sample.
	\end{itemize}
\end{itemize}
\end{frame}

\begin{frame}{Other Issues}
\begin{itemize}
	\item<+-> \textbf{Small sample sizes}
	\begin{itemize}
		\item<+-> Make sure the sample size is large enough to include a variety of data regarding the population.
		\item<+-> More info on this later in the course.
	\end{itemize}
	\item<+-> \textbf{Non-responses}
	\begin{itemize}
		\item<+-> When someone refuses to respond to a question or is unavailable.
		\item<+-> How are missing values handled? As n/a? As 0?
	\end{itemize}
	\item<+-> \textbf{Loaded question}
	\begin{itemize}
		\item<+-> A question worded in order to mislead or elicit a desired response.
	\end{itemize}
\end{itemize}
\end{frame}

\begin{frame}{Example 3}
Which of the following is a loaded question?	\newline\\
(1) \quad ``Should the fire department receive additional funding for new equipment?"	\newline\\
(2) \quad ``Should taxpayers be responsible for new fire department equipment?"			\newline\\ \pause

Statement (2) is a loaded question.
\end{frame}

\end{document}