\documentclass[t]{beamer}
\usetheme{Copenhagen}
\setbeamertemplate{headline}{} % remove toc from headers
\beamertemplatenavigationsymbolsempty

\usepackage{amsmath, array, tikz, bm, pgfplots, tcolorbox, graphicx, venndiagram, color, colortbl, xfrac}
\pgfplotsset{compat = 1.16}
\usepgfplotslibrary{statistics}
\usetikzlibrary{calc}

\title{Standard Normal Distribution}
\author{}
\date{}

\AtBeginSection[]
{
  \begin{frame}
    \frametitle{Objectives}
    \tableofcontents[currentsection]
  \end{frame}
}

\begin{document}

\pgfmathdeclarefunction{gauss}{2}{%
  \pgfmathparse{1/(#2*sqrt(2*pi))*exp(-((x-#1)^2)/(2*#2^2))}%
}

\begin{frame} 
\maketitle
\end{frame}

\section{Find the area under a normal curve given z score(s)}

\begin{frame}{Continuous Distributions}
\begin{tcolorbox}[colframe=green!20!black, colback = green!30!white,title=\textbf{Continuous Probability Distribution}]
A \textbf{continuous probability distribution} is a probability distribution in which the observations are continuous variables.
\end{tcolorbox}
\bigskip	\pause

In this section, we are going to discuss the {\color{blue}\textbf{standard normal distribution}}, whose histogram resembles a bell-shaped curve.
\end{frame}

\begin{frame}{Equation and Graph of Standard Normal Distribution}
\[f(z) = \frac{1}{\sqrt{2\pi}} e^{-z^2/2} \]	\smallskip
\begin{center}
\begin{tikzpicture}[scale=0.7]
\begin{axis}[no marks, samples=200, axis lines = left, xmin = -3.5, xmax = 3.5, ymax = 0.45,
xlabel = {z score}, ylabel = {Probability}
]
\addplot [very thick, color=blue] {gauss(0,1)};
\end{axis}
\end{tikzpicture}
\end{center}
\end{frame}

\begin{frame}{Properties of the Standard Normal Distribution}
\begin{itemize}
	\item<+-> The mean is 0 and the standard deviation is 1
	\item<+-> The graph is symmetric about the mean
	\item<+-> The area under the curve represents the probability of obtaining a z score in that area.
	\item<+-> The total area under the curve equals 1
\end{itemize}
\end{frame}

\begin{frame}{Example 1}
Find the area under the curve to the left of each of the following z scores.	\newline\\	
(a)	\quad	$z = 0$	\newline\\
\begin{minipage}{0.5\textwidth}
\onslide<2->{
\begin{tikzpicture}[scale=0.7]
\begin{axis}[no marks, samples=200, axis lines = left, xmin = -3.5, xmax = 3.5, ymax = 0.45,
xlabel = {z score}, ylabel = {Probability}
]
\addplot [draw = none, fill = green, domain = -3.5:0] {gauss(0,1)} \closedcycle;
\addplot [very thick, color=blue] {gauss(0,1)};
\end{axis}
\end{tikzpicture}}
\end{minipage}
\hspace{0.75cm}
\begin{minipage}{0.25\textwidth}
\onslide<3->{0.5}	\newline\\
\onslide<4->{\[P(z \leq 0) = 0.5\]}
\end{minipage}
\end{frame}

\begin{frame}{Example 1}
(b)	\quad	$z = -1$	\newline\\
\begin{minipage}{0.5\textwidth}
\begin{tikzpicture}[scale=0.7]
\begin{axis}[no marks, samples=200, axis lines = left, xmin = -3.5, xmax = 3.5, ymax = 0.45,
xlabel = {z score}, ylabel = {Probability}
]
\addplot [draw = none, fill = green, domain = -3.5:-1] {gauss(0,1)} \closedcycle;
\addplot [very thick, color=blue] {gauss(0,1)};
\end{axis}
\end{tikzpicture}
\end{minipage}
\hspace{0.75cm}
\begin{minipage}{0.25\textwidth}
\onslide<2->{0.1587}	\newline\\
\onslide<3->{$P(z \leq -1) = 0.1587$}
\end{minipage}
\end{frame}

\begin{frame}{Example 1}
(c)	\quad	$z = 1.75$	\newline\\
\begin{minipage}{0.5\textwidth}
\begin{tikzpicture}[scale=0.7]
\begin{axis}[no marks, samples=200, axis lines = left, xmin = -3.5, xmax = 3.5, ymax = 0.45,
xlabel = {z score}, ylabel = {Probability}
]
\addplot [draw = none, fill = green, domain = -3.5:1.75] {gauss(0,1)} \closedcycle;
\addplot [very thick, color=blue] {gauss(0,1)};
\end{axis}
\end{tikzpicture}
\end{minipage}
\hspace{0.75cm}
\begin{minipage}{0.25\textwidth}
\onslide<2->{0.9599}	\newline\\
\onslide<3->{$P(z \leq 1.75) = 0.9599$}
\end{minipage}
\end{frame}

\begin{frame}{Example 2}
Find the area under the curve to the right of each of the following z scores.	\newline\\	
(a)	\quad	$z = -0.25$	\newline\\
\begin{minipage}{0.5\textwidth}
\onslide<2->{
\begin{tikzpicture}[scale=0.7]
\begin{axis}[no marks, samples=200, axis lines = left, xmin = -3.5, xmax = 3.5, ymax = 0.45,
xlabel = {z score}, ylabel = {Probability}
]
\addplot [draw = none, fill = green, domain = -0.25:3.5] {gauss(0,1)} \closedcycle;
\addplot [very thick, color=blue] {gauss(0,1)};
\end{axis}
\end{tikzpicture}}
\end{minipage}
\hspace{0.75cm}
\begin{minipage}{0.25\textwidth}
\onslide<3->{0.5987}	\newline\\
\onslide<4->{$P(z \geq -0.25) = 0.5987$}
\end{minipage}
\end{frame}

\begin{frame}{Example 2}
(b)	\quad	$z = 2$	\newline\\
\begin{minipage}{0.5\textwidth}
\onslide<2->{
\begin{tikzpicture}[scale=0.7]
\begin{axis}[no marks, samples=200, axis lines = left, xmin = -3.5, xmax = 3.5, ymax = 0.45,
xlabel = {z score}, ylabel = {Probability}
]
\addplot [draw = none, fill = green, domain = 2:3.5] {gauss(0,1)} \closedcycle;
\addplot [very thick, color=blue] {gauss(0,1)};
\end{axis}
\end{tikzpicture}}
\end{minipage}
\hspace{0.75cm}
\begin{minipage}{0.25\textwidth}
\onslide<3->{0.0228}	\newline\\
\onslide<4->{$P(z \geq 2) = 0.0228$}
\end{minipage}
\end{frame}

\begin{frame}[c]{Finding the Area Between Two z Scores}
\begin{center}
\scalebox{0.5}{
\begin{tikzpicture}[scale=0.8]
\begin{axis}[no marks, samples=200, axis lines = left, xmin = -3.5, xmax = 3.5, ymax = 0.45,
xlabel = {z score}, ylabel = {Probability}
]
\addplot [draw = none, fill = green, domain = -3.5:2] {gauss(0,1)} \closedcycle;
\addplot [very thick, color=blue] {gauss(0,1)};
\end{axis}
\end{tikzpicture}
\hspace{0.2cm} \raisebox{2.5cm}{\Huge{$-$}} \hspace{0.2cm}
\begin{tikzpicture}[scale=0.8]
\begin{axis}[no marks, samples=200, axis lines = left, xmin = -3.5, xmax = 3.5, ymax = 0.45,
xlabel = {z score}, ylabel = {Probability}
]
\addplot [draw = none, fill = green, domain = -3.5:-2] {gauss(0,1)} \closedcycle;
\addplot [very thick, color=blue] {gauss(0,1)};
\end{axis}
\end{tikzpicture}
\hspace{0.2cm} \raisebox{2.5cm}{\Huge{$=$}} \hspace{0.2cm}
\begin{tikzpicture}[scale=0.8]
\begin{axis}[no marks, samples=200, axis lines = left, xmin = -3.5, xmax = 3.5, ymax = 0.45,
xlabel = {z score}, ylabel = {Probability}
]
\addplot [draw = none, fill = green, domain = -2:2] {gauss(0,1)} \closedcycle;
\addplot [very thick, color=blue] {gauss(0,1)};
\end{axis}
\end{tikzpicture}}
\end{center}
\end{frame}

% Standard Normal Distribution
%  - Equation and Graph
%  - Mean & SD
%  - Properties
%  - Finding area under curve
%    * Left/Right/Between/Outside

\section{Find the z scores corresponding to a given area}
%  - Finding z scores for a given area
%    * Left/Right/Between/Outside

\end{document}