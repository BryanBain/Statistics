\documentclass[t]{beamer}
\usetheme{Copenhagen}
\setbeamertemplate{headline}{} % remove toc from headers
\beamertemplatenavigationsymbolsempty

\usepackage{amsmath, array, tikz, bm, pgfplots, tcolorbox, graphicx, venndiagram, color, colortbl, xfrac}
\pgfplotsset{compat = 1.16}
\usepgfplotslibrary{statistics}
\usetikzlibrary{calc}

\title{Confidence Intervals}
\author{}
\date{}

\AtBeginSection[]
{
  \begin{frame}
    \frametitle{Objectives}
    \tableofcontents[currentsection]
  \end{frame}
}

\begin{document}

\pgfmathdeclarefunction{gauss}{2}{%
  \pgfmathparse{1/(#2*sqrt(2*pi))*exp(-((x-#1)^2)/(2*#2^2))}%
}

\begin{frame} 
\maketitle
\end{frame}

\section{Determine confidence intervals for population mean}

\begin{frame}{How Close Are We to the Population Mean?}
In the last section, we looked at sampling distributions of the sample mean (and proportion too). Throughout the series, we've used computer simulations to examine statistical concepts. \newline\\	\pause

However, in real life, there are factors that can limit the number of studies and samples we can take: time, cost, etc. \newline\\	\pause

So, for our samples, how confident are we that they contain the population mean?	\newline\\	\pause

That is where confidence intervals come into play.
\end{frame}

\begin{frame}{How Confident Are We?}
\begin{tcolorbox}[colframe=green!20!black, colback = green!30!white,title=\textbf{Confidence Interval}]
A \textbf{confidence interval} for a population parameter is an estimate of possible values for the parameter with a \emph{given} certain level of confidence.
\end{tcolorbox}
\bigskip \pause

\begin{tcolorbox}[colframe=green!20!black, colback = green!30!white,title=\textbf{Confidence Level}]
The \textbf{confidence level}, or \textbf{level of confidence}, is the percentage of the number of times our confidence intervals will contain the population parameter.
\end{tcolorbox}
\bigskip \pause

Typical confidence levels are 90\%, 95\%, 98\%, and 99\%.
\end{frame}

\begin{frame}{Confidence Interval Setup}
A confidence interval for a population parameter is in the form
\[\text{point estimate} \pm \text{margin of error}\]	\pause

\begin{tcolorbox}[colframe=green!20!black, colback = green!30!white,title=\textbf{Point Estimate}]
A \textbf{point estimate} is a value based on our sample data that represents a reasonable value of the population parameter.
\end{tcolorbox}
\bigskip \pause

The margin of error is in the form
\begin{center}
critical value $\times$ standard error
\end{center}
\end{frame}

\begin{frame}{Critical Values}
Critical values are typically in the form $z_{\alpha/2}$ where	
\begin{center}
\begin{tikzpicture}[scale=0.8]
\begin{axis}[
axis lines = left, xmin = -3.5, xmax = 3.5, ymin = 0, ymax = 0.45, xticklabels = {},
xtick style={draw=none}, xlabel = {z Score}, ylabel = {Probability}
]
\addplot [draw = none, fill = green, domain = -3.5:-1.5] {gauss(0,1)} \closedcycle;
\addplot [draw = none, fill = green, domain = 1.5:3.5] {gauss(0,1)} \closedcycle;
\addplot [color=blue, samples=200,thick, domain=-3.5:3.5] {gauss(0,1)};
\draw [<->,>=stealth,thick] (axis cs: -1.5,0.05) -- (axis cs: 1.5,0.05) node [above, midway] {$\bm{1-\alpha}$};
\draw [->,>=stealth,thick] (axis cs: -2.65,0.1) node [above] {$\bm{\alpha/2}$} -- (axis cs: -2.05,0.015);
\draw [->,>=stealth,thick] (axis cs: 2.65,0.1) node [above] {$\bm{\alpha/2}$} -- (axis cs: 2.05,0.015);
\end{axis}
\end{tikzpicture}
\end{center}
\end{frame}

\begin{frame}{Common Critical Values of $z_{\alpha/2}$}
\begin{center}
\begin{tabular}{c|c}
Confidence Level & Critical Value of $z_{\alpha/2}$ \\ \hline
\onslide<2->{90\%} & \onslide<2->{$\pm 1.645$} \\[8pt]
\onslide<3->{95\%} & \onslide<3->{$\pm 1.96$} \\[8pt]
\onslide<4->{98\%} & \onslide<4->{$\pm 2.326$} \\[8pt]
\onslide<5->{99\%} & \onslide<5->{$\pm 2.576$} \\[8pt]
\end{tabular}
\end{center}
\onslide<6->{Notice the higher the confidence level, the further away from $z = 0$ the critical value is.}	\newline\\
\onslide<7->{The standard error is still $\frac{\sigma}{\sqrt{n}}$}
\end{frame}

\begin{frame}{Example 1}
The waiting time of a sample of 100 patients at a hospital is 3.5 minutes. Construct a 95\% confidence interval for the mean waiting time of all patients at the hospital. Assume $\sigma = 0.75$ minutes.	\newline\\	\pause
\end{frame}

\begin{frame}{Error Bars}
Often times, margins of error are graphed as \textit{error bars} on a visual display. \newline\\	\pause
\begin{center}
% TODO: Draw a bar chart with error bars
\end{center}
However, be advised that some graphs use standard deviation for their error bars.
\end{frame}
%  - Margin of Error and Error Bars
\section{Determine confidence intervals for population proportion}

\section{Determine the necessary sample size}


\end{document}