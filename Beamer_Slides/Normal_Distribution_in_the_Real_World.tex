\documentclass[t]{beamer}
\usetheme{Copenhagen}
\setbeamertemplate{headline}{} % remove toc from headers
\beamertemplatenavigationsymbolsempty

\usepackage{amsmath, array, tikz, bm, pgfplots, tcolorbox, graphicx, venndiagram, color, colortbl, xfrac}
\pgfplotsset{compat = 1.16}
\usepgfplotslibrary{statistics}
\usetikzlibrary{calc}

\title{Normal Distribution in the Real World}
\author{}
\date{}

\AtBeginSection[]
{
  \begin{frame}
    \frametitle{Objectives}
    \tableofcontents[currentsection]
  \end{frame}
}

\begin{document}

\pgfmathdeclarefunction{gauss}{2}{%
  \pgfmathparse{1/(#2*sqrt(2*pi))*exp(-((x-#1)^2)/(2*#2^2))}%
}

\begin{frame} 
\maketitle
\end{frame}

\section{Find the area under a normal curve for an observed value}

\begin{frame}{Normal Distribution Revisited}
The equation for calculating area when working with z scores is
\[f(z) = \frac{1}{\sqrt{2\pi}} e^{-z^2/2}\]
with mean $\mu = 0$ and standard deviation $\sigma = 1$.	\newline\\	\pause

Using observed values, and the z score formula $z = \frac{x-\mu}{\sigma}$, we get the following equation in terms of an observed value, $x$:	\pause
\[f(x) = \frac{1}{\sigma\sqrt{2\pi}} e^{-\left(\frac{x-\mu}{\sigma}\right)^2/2\sigma^2}\]
\end{frame}

\begin{frame}[c]{Observed Value Graph}
\begin{center}
\begin{tikzpicture}[scale=0.7]
\begin{axis}[no marks, samples=200, axis lines = left, xmin = -3.5, xmax = 3.5, ymax = 0.45, xscale=1.5,
xlabel = {observed values}, ylabel = {Probability}, xtick = {-3,-2,...,3}, xticklabels = {$\mu-3\sigma$,$\mu-2\sigma$,$\mu-\sigma$, $\mu$, $\mu+\sigma$, $\mu+2\sigma$, $\mu+3\sigma$} 
]
\addplot [very thick, color=blue] {gauss(0,1)};
\end{axis}
\end{tikzpicture}
\end{center}
\end{frame}

\begin{frame}{Which Method Should I Use?}
You have the option to convert your observed value to a z score using the given mean and standard deviation.	\newline\\	\pause

However, most technology allows you to enter the observed value, mean, and standard deviation without having to convert to a z score first.	\newline\\	\pause

Go with what you feel most comfortable with.
\end{frame}

\begin{frame}{Example 1}
(a) \quad An air conditioning manufacturer says their machines have a mean cooling capacity of 10,000 BTUs. Due to variations that arise in manufacturing, there is a standard deviation of 400 BTUs. 	\newline\\ What is the probability of selecting an air conditioning unit that is 11,000 BTUs or less?	\newline\\	\pause

\begin{center}
\begin{tikzpicture}[scale=0.55]
\begin{axis}[no marks, samples=200, axis lines = left, xmin = -3.5, xmax = 3.5, ymax = 0.45, xscale=1.5,
xlabel = {observed values}, ylabel = {Probability}, xtick = {-3,-2,...,3}, xticklabels = {$8800$, $9200$, $9600$, $10000$, $10400$, $10800$, $11200$} 
]
\addplot [draw = none, fill = green, domain = -3.5:2.5] {gauss(0,1)} \closedcycle;
\addplot [very thick, color=blue] {gauss(0,1)};
\end{axis}
\end{tikzpicture}
\end{center}
\end{frame}

\begin{frame}{Example 1a}
We have $x = 11,000, \, \mu = 10,000, \, \text{and } \sigma = 400$	\newline\\	\pause
We are finding the area \emph{to the left of} $x = 11,000$:	\pause
\[x \approx 0.9938\]	\pause
There is about a 99.38\% chance of selecting an air conditioning unit that is 11,000 BTUs or less.
\end{frame}

\begin{frame}{Example 1}
(b) \quad What is the probability that you select an air conditioning unit between 9,500 and 10,600 BTUs?	\newline\\	\pause

\begin{center}
\begin{tikzpicture}[scale=0.65]
\begin{axis}[no marks, samples=200, axis lines = left, xmin = -3.5, xmax = 3.5, ymax = 0.45, xscale=1.5,
xlabel = {observed values}, ylabel = {Probability}, xtick = {-3,-2,...,3}, xticklabels = {$8800$, $9200$, $9600$, $10000$, $10400$, $10800$, $11200$} 
]
\addplot [draw = none, fill = green, domain = -1.25:1.5] {gauss(0,1)} \closedcycle;
\addplot [very thick, color=blue] {gauss(0,1)};
\end{axis}
\end{tikzpicture}
\end{center}
\end{frame}

\begin{frame}{Example 1b}
We have $x_1 = 9,500, \, x_2 = 10,600, \, \mu = 10,000, \, \text{and } \sigma = 400$	\newline\\	\pause
We are finding the area \emph{between} $x = 9,500$ and $x = 10,600$:	\pause
\[x \approx 0.8275\]	\pause
There is about an 82.75\% chance of selecting an air conditioning unit between 9,500 and 10,600 BTUs.
\end{frame}

\section{Find observed values given area under the normal curve}

\begin{frame}{Finding Values as the Inverse}
Once again, this is just the inverse of what we did in the last example.	\newline\\	\pause

Fortunately, most statistical technology has the ability to solve problems like these without the need to convert back from z scores first.
\end{frame}

\begin{frame}{Example 2}
(a) \quad An air conditioning manufacturer says their machines have a mean cooling capacity of 10,000 BTUs. Due to variations that arise in manufacturing, there is a standard deviation of 400 BTUs. 	\newline\\ How many BTUs must an air conditioning unit have to be in the 90th percentile?	\newline\\	\pause

The 90th percentile means that 90\% of the other units have less BTUs than the value we are finding.	
\end{frame}

\begin{frame}{Example 2a}
\begin{center}
\begin{tikzpicture}[scale=0.55]
\begin{axis}[no marks, samples=200, axis lines = left, xmin = -3.5, xmax = 3.5, ymax = 0.45, xscale=1.5,
xlabel = {observed values}, ylabel = {Probability}, xtick = {-3,-2,...,3}, xticklabels = {$8800$, $9200$, $9600$, $10000$, $10400$, $10800$, $11200$} 
]
\addplot [draw = none, fill = green, domain = -3.5:1.282] {gauss(0,1)} \closedcycle;
\addplot [very thick, color=blue] {gauss(0,1)};
\node at (axis cs: 0,0.15) {$\bm{0.9}$};
\end{axis}
\end{tikzpicture}
\end{center}
\pause
\[x \approx 10,512.62\]	\pause
An air conditioning unit with about 10,512 BTUs is in the 90th percentile.
\end{frame}

\begin{frame}{Example 2}
(b) \quad An air conditioning unit will not be shipped if it is in either the bottom 0.25\% or top 0.25\% of BTUs. What are those BTU cutoff values?	\newline\\	\pause

\begin{center}
\begin{tikzpicture}[scale=0.55]
\begin{axis}[no marks, samples=200, axis lines = left, xmin = -3.5, xmax = 3.5, ymax = 0.45, xscale=1.75, yscale=1.5,
xlabel = {observed values}, ylabel = {Probability}, xtick = {-3,-2,...,3}, xticklabels = {$8800$, $9200$, $9600$, $10000$, $10400$, $10800$, $11200$} 
]
\addplot [draw = none, fill = red, domain = -3.5:-2.807] {gauss(0,1)} \closedcycle;
\addplot [draw = none, fill = red, domain = 2.807:3.5] {gauss(0,1)} \closedcycle;
\addplot [very thick, color=blue] {gauss(0,1)};
\end{axis}
\end{tikzpicture}
\end{center}
\end{frame}

\begin{frame}{Example 2b}
So, we are looking to find the BTU values that 99.5\% of the air conditioning units will fall between.	\newline\\	\pause
\[x_1 \approx 8,877.19 \text{ and } x_2 \approx 11,122.81\]	\newline\\	\pause

An air conditioning unit will not be shipped if it is between around 8,877 and 11,123 BTUs.
\end{frame}

\end{document}