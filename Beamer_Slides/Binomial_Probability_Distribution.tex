\documentclass[t]{beamer}
\usetheme{Copenhagen}
\setbeamertemplate{headline}{} % remove toc from headers
\beamertemplatenavigationsymbolsempty

\usepackage{amsmath, array, tikz, bm, pgfplots, tcolorbox, graphicx, venndiagram, color, colortbl, xfrac}
\pgfplotsset{compat = 1.16}
\usepgfplotslibrary{statistics}
\usetikzlibrary{calc}

\title{Binomial Probability Distributions}
\author{}
\date{}

\AtBeginSection[]
{
  \begin{frame}
    \frametitle{Objectives}
    \tableofcontents[currentsection]
  \end{frame}
}

\begin{document}

\begin{frame} 
\maketitle
\end{frame}

\section{Calculate probabilities of binomial distributions}

\begin{frame}{Binomial Probability Experiment}
A {\color{blue}\textbf{binomial probability experiment}} is an experiment with the following conditions:
	\begin{itemize}
		\item<2-> There are a fixed number of $n$ repeated independent trials
		\item<3-> Each trial's outcome is either a success or failure
		\item<4-> The probability of success, $p$, never changes
	\end{itemize}
\end{frame}

\begin{frame}{Flipping a Coin}
Number of heads when flipping a coin 3 times:
\begin{center}
\setlength{\extrarowheight}{4pt}
\begin{tabular}{c|c|c}
$\bm{x}$ & Outcomes &$\bm{P(X=x)}$ \\ \hline
0 & TTT & 1/8 \\[4pt]
1 & HTT THT TTH & 3/8 \\[4pt]
2 & HHT HTH THH & 3/8 \\[4pt]
3 & HHH & 1/8	\\[8pt]
\end{tabular}
\end{center}
\onslide<2->{In the outcome for 2 heads, HHT, HTH, and THH are all considered equal.}	\newline\\
\onslide<3->{Thus, combinations play a role in binomial probability distributions.}
\end{frame}

\begin{frame}{Flipping a Coin}
Also note that if two heads are obtained when flipping a coin 3 times, then the remaining flips (1 in this case) must be tails. \newline\\	\pause

Heads has a probability of $\frac{1}{2}$ of occurring and tails has a probability of $1 - \frac{1}{2}$ of occurring.	\newline\\	\pause

Combining the idea of combinations, the number of successes of each event, and the probability of success and failure, we get the formula for calculating probabilities for binomial distributions:	\pause
\begin{align*}
\onslide<4->{P(X=x) &= \binom{n}{x} \cdot p^x \cdot (1-p)^{n-x}}	\\[5pt]
\onslide<5->{&= \frac{n!}{x!(n-x)!} \cdot p^x \cdot (1-p)^{n-x}}
\end{align*}
\end{frame}

\begin{frame}{Example 1}
Which is more likely when flipping a coin: obtaining 4 out of 5 tails or obtaining 8 out of 10 tails?	\newline\\	\pause
4 out of 5:	\pause
\[\frac{5!}{4!(5-4)!}\left(\frac{1}{2}\right)^4 \left(\frac{1}{2}\right)^1 \onslide<3->{= 0.15625}\] 	

\onslide<4->{8 out of 10:} 
\onslide<5->{\[\frac{10!}{8!(10-8)!}\left(\frac{1}{2}\right)^8 \left(\frac{1}{2}\right)^2 \onslide<6->{\approx 0.04395}\]}
\onslide<7->{It is more likely you will obtain 4 heads from 5 flips than 8 heads from 10 flips.}
\end{frame}

\begin{frame}{Example 2}
(a) \quad A multiple choice test consists of 10 questions with 5 answer choices per question. What is the probability that a student guesses correctly on 6 questions?	

\onslide<2->{
	\begin{align*}
	P(X=6) &= \frac{10!}{6!(10-6)!}\left(\frac{1}{5}\right)^6 \left(\frac{4}{5}\right)^4 \\[12pt]
	\onslide<3->{&\approx 0.00551}
	\end{align*}
}
\end{frame}

\begin{frame}{Example 2}
(b) \quad What is the probability a student guesses \emph{at least} 6 out of 10 questions correctly?	\newline\\

\onslide<2->{We can either add the probabilities from $x = 6$ to $x = 10$, or use a cumulative binomial calculation with the {\color{blue}\textbf{complement rule}} to achieve this as well.}

\begin{align*}
\onslide<3->{P(x \geq 6) &= 1 - P(x < 6)} \\[6pt]
\onslide<4->{& \approx 0.00637}
\end{align*}
\end{frame}

\section{Calculate the mean, variance, and standard deviation of a binomial distribution}

% Fixed number of independent trials
% Each trial = success or failure
% p never changes
% nCr * p^x * (1-p)^(n-k)

% Mean Var & SD

\end{document}